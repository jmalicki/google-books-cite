\documentclass{article}
\usepackage[margin=1in]{geometry}
\usepackage{google-books-cite}
\usepackage[style=apa,backend=biber]{biblatex}
\addbibresource{example.bib}

% Register Google Books IDs for your references
\SetGoogleBooksID{Schopenhauer1969}{NbsVAAAAYAAJ}
\SetGoogleBooksID{Kant1785}{L9ATAAAAQAAJ}

% Optional: customize appearance
\SetGBIndicator{superscript}  % or 'brackets' or 'none'
\SetGBLinkColor{DarkCyan}

\title{Example: Google Books Citations}
\author{Your Name}

\begin{document}
\maketitle

\section{Introduction}

This document demonstrates the \texttt{google-books-cite} package.
\gbdisclaimer  % Adds explanatory footnote about GB links

\section{Examples}

% Regular citation (no Google Books ID registered):
As noted in recent scholarship \parencite[p.~42]{ModernWork2020}, the issue remains debated.

% Google Books citation (page number becomes clickable):
Schopenhauer argues that satisfaction ``at once makes room for a new one'' 
\gbparencite[p.~312]{Schopenhauer1969}, creating an endless cycle.

% Text citation variant:
\gbtextcite[p.~27]{Kant1785} famously defines the good will as good in itself.

% Multiple pages (links to first page):
The argument develops across several sections 
\gbparencite[pp.~312--319]{Schopenhauer1969}.

% No page number (behaves like regular citation):
This theme recurs throughout \gbparencite{Schopenhauer1969}.

\section{Visual Indicators}

Notice that Google Books links are marked with $^{\text{\tiny GB}}$ and 
appear in a distinct color (teal by default). Regular citations remain 
black or the default link color.

\printbibliography

\end{document}

